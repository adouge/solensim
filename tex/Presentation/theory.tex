\tocslide
\subsection{General notions}
\begin{frame}
  \frametitle{Electron optics}
  \rfn
  \framesubtitle{Definition}
  \begin{table}[t]
    \centering
    \begin{tabular}{   m{5cm}  m{5cm}  }
    \underline{Classic optics} & \underline{Electron optics} \\
    & \\
    \bull Light ray - photons &  \bull Electron beam - electrons  \\
    & \\
    \bull Can pass through optically transparent solids  & \bull  Gets absorbed/loses energy due to interactions with atoms in media \\
    & \\
    \bull Bends due to refractive index difference between media & \bull  Bends due to Coulomb and Lorentz forces in the presense of external EM-fields \\
    \end{tabular}
  \end{table}
\end{frame}

\begin{frame}
  \frametitle{Magnet lenses}
  \framesubtitle{Requirements}
  \rfn
  A lens must have the following properties:
  \vspace{0.15cm}
  \begin{itemize}
    \item deflection increases with increasing deviation of the beam from the optic axis;
    \vspace{0.15cm}
    \item electron energy should not change, or change negligebly;
    \vspace{0.15cm}
    \item symmetry of deflection on all sides of the optical axis;
    \vspace{0.15cm}
    \item methods and laws of classical optics (such as thin lens formula and approximation, matrix formalism) are assumed to be applicable.
  \end{itemize}
\end{frame}

\begin{frame}
  \frametitle{Magnet lenses}
  \framesubtitle{Solenoids}
  \begin{figure}
    \subfloat[A solenoid with S as optical axis\footcite{a_optics}]{\includegraphics[width=0.5\textwidth]{Solenoid_scheme3.png}}
    \subfloat[Cross-section of a magnetic lens\footcite{Egerton}]{\includegraphics[width=0.5\textwidth]{Solenoid_pic1.png}}
    \label{some example}
  \end{figure}
\end{frame}

\begin{frame}
  \frametitle{Magnet lenses}
  \framesubtitle{Solenoids}
  \rfn
  \begin{figure}
    \includegraphics[width=.5\textwidth]{El_traj_in_B1.png}
    \caption{A solenoid cross-section along the optic axis\footcite{Egerton}}
  \end{figure}
  \begin{scriptsize}
  \begin{columns}
    \begin{column}{0.45\textwidth}
      \begin{equation}
        F_{\varphi}=e\left(B_{z}v_{r} - v_{z}B_{r}\right)
      \end{equation}
      \begin{equation}
        F_{r}=-e\left(v_{z}B_{z}\right)
      \end{equation}
      \begin{equation}
        F_{z}=e\left(v_{\varphi}B_{r}\right)
      \end{equation}
    \end{column}
    \begin{column}{0.55\textwidth}
      \begin{equation}
        \gamma m\left(\ddot{r}-r\dot{\varphi}^{2}\right)=-er\dot{\varphi}B_{z}
      \end{equation}
      \begin{equation}
        \gamma m\frac{d}{dt}\left(r^{2}\dot{\varphi}\right)=-er(\dot rB_{z}+B_{r}\dot{z})
      \end{equation}
      \begin{equation}
        \gamma m\ddot{z}=er\dot{\varphi}B_{r}
      \end{equation}
    \end{column}
  \end{columns}
  \end{scriptsize}
\end{frame}

\begin{frame}
  \frametitle{Magnet lenses}
  \framesubtitle{Electron path equations}
  \rfn
  \begin{scriptsize}
  \begin{columns}
    \begin{column}{0.6\textwidth}
      \begin{equation}
        B_{z}\left(z,r\right)=\underset{n}{\sum}\frac{\left(-1\right)^{n}}{n!n!}\left(\frac{r}{2}\right)^{2n}\frac{\partial^{2n}B_{z,\,axis}}{\partial z^{2n}}
      \end{equation}
      \begin{equation}
        B_{r}\left(z,r\right)=\underset{n}{\sum}\frac{\left(-1\right)^{n}}{n!\left(n-1\right)!}\left(\frac{r}{2}\right)^{2n-1}\frac{\partial^{2n-1}B_{z,\,axis}}{\partial z^{2n-1}}
      \end{equation}
    \end{column}
    \begin{column}{0.5\textwidth}
      \bull From (5) follows:
      \begin{equation}
        \dot{\varphi}=\frac{e}{2\gamma m}B_{z}
      \end{equation}
      \bull From (4) and (6) follows:
      \begin{equation}
        \ddot{r}=-\left(\frac{e}{2\gamma m}\right)^{2}rB_{z}^{2}
      \end{equation}
      \begin{equation}
        \ddot{z}=-\left(\frac{e}{2\gamma m}\right)^{2}r^{2}B_{z}B'_{z}
      \end{equation}
    \end{column}
  \end{columns}
  \begin{equation}
    \ddot{r}=r''\dot{z}^{2}\approx r''\left(\beta c\right)^{2}\Rightarrow r''=\left(\frac{e}{2p_{z}}\right)^{2}rB_{z}^{2}
  \end{equation}
  \begin{equation}
    -\frac{r'}{r}=\frac{1}{f}\underset{integrate\,(12)}{\Rightarrow}\frac{1}{f}=\left(\frac{e}{2p_{z}}\right)^{2}\intop_{-\infty}^{\infty}B_{z}^{2}dz\coloneqq\left(\frac{e}{2p_{z}}\right)^{2}F_{2}
  \end{equation}
  \end{scriptsize}
\end{frame}

\begin{frame}
  \frametitle{Magnet lenses}
  \framesubtitle{Magnetic field of solenoid}
  \rfn
  \vspace{-1cm}
  \begin{figure}
    \centering
    \subfloat[Single-wind solenoid\footcite{Disser}]{\includegraphics[width = 0.3\textwidth]     {Single_lay_solenoid2.png}}
    \hspace{0.1cm}
    \subfloat[Solenoid with multilayred windings\citesame]{\includegraphics[width = 0.3\textwidth]{Multilay_solenoid2.png}}
    %\hspace{0.1cm}
    \subfloat[Fields produced by two coils with same N and L=b]{\includegraphics[width = 0.38\textwidth]{fieldcomparisson.png}}
    \label{some example}
  \end{figure}
  \begin{tiny}
  \begin{columns}
    \begin{column}{0.5\textwidth}
      Field of the solenoid (a):
      \begin{equation}
      B_{z}\!\left(z\right)=\!\frac{\mu_{0}nI}{2}\!\left(\!\frac{\triangle z}{\sqrt{R^{2}+\triangle z^{2}}}\!-\!\frac{\triangle z^{*}}{\sqrt{R^{2}+\triangle z^{*2}}}\!\right)\!
      \end{equation}
      \begin{equation}
      \triangle z=z-L/2
      \end{equation}
    \end{column}
    \begin{column}{0.6\textwidth}
      Approximate field of the solenoid (b)\citesame:
      \begin{equation}
        B_{z}\left(z\right)\approx\frac{\mu_{0}NI}{4}\left(\frac{Rc^{2}}{\left(z^{2}+Rc^{2}\right)^{3/2}}+\frac{Rc^{*2}}{\left(z^{2}+Rc^{*2}\right)^{3/2}}\right)
      \end{equation}
      \begin{equation}
        Rc=R_{sq}+c,\,R_{sq}=R_{m}\left(1+\frac{a^{2}}{24R_{m}^{2}}\right),\,c^{2}=\frac{b^{2}-a^{2}}{12}
      \end{equation}
    \end{column}
  \end{columns}
  \end{tiny}
\end{frame}

\subsection{Lens imperfections}

\begin{frame}
  \frametitle{Lens impefections}
  \rfn
  %\framesubtitle{Defects}
  \begin{small}
  There are 3 main limitations to consider when designing a solenoid lens:
  \begin{table}[t]
    \centering
    \begin{tabular}{   m{5cm}  m{5cm}  }
      &  \underline{Source:}\vspace{0.2cm} \\
      \bull Chromatic aberration & Spread of electron energies  \\
      & \\
      \bull RMS Emittance growth  &   Spread of eletron coordinates in position-and-momentum phase space \\
      & \\
      \bull Spherical aberration &  Real lens $\rightarrow$ different refraction based on distance from axis \\
    \end{tabular}
  \end{table}
  \end{small}
  \vspace{1cm}
  \begin{tiny}
  Emittance:
  \begin{equation}
    \epsilon_{n,rms}=\frac{1}{mc}\sqrt{\left\langle x^{2}\right\rangle \left\langle \tilde{p}_{x}^{2}\right\rangle -\left\langle x\tilde{p}_{x}\right\rangle ^{2}}=\frac{1}{mc}\left(\frac{e^{2}\sigma^{4}}{3\sqrt{2}p_{z,0}}F_{3}+\frac{e^{4}\sigma^{4}}{24\sqrt{2}p_{z,0}^{3}}F_{4}\right)
  \end{equation}
  Chromatic aberration:
  \begin{equation}
    r_{c}\approx\alpha f\left(\triangle E_{0}/E_{0}\right)\approx\alpha C_{c}\left(\triangle E_{0}/E_{0}\right)
  \end{equation}
  \end{tiny}
\end{frame}

\begin{frame}
  \frametitle{Lens impefections}
  \framesubtitle{Spherical aberration\footcite{Egerton}}
  \rfn
  \begin{columns}
    \begin{column}{0.4\textwidth}
      \begin{tiny}
      \begin{equation}
        \triangle f=c\cdotp x^{2}
      \end{equation}
      \begin{equation}
        x\approx f\cdotp tan\left(\alpha\right)
      \end{equation}
      \begin{align}
        r_{s}&=\triangle f\cdotp tan\left(\alpha\right)\approx c\cdotp f^{2}tan\left(\alpha\right)^{3}\\
        &=C_{s}\cdotp\left(\frac{r_{in}}{f-\frac{C_{s}r_{in}^{2}}{f^{2}}}\right)^{3}
      \end{align}
      \begin{equation}
        F_{3}=-\int\frac{B_{z}B''_{z}}{2}dz
      \end{equation}
      \begin{equation}
        F_{4}=\int B_{z}^{4}dz
      \end{equation}
      \end{tiny}
    \end{column}
    \begin{column}{0.6\textwidth}
    \begin{figure}[R]
      %\vspace{-1.5cm}\hspace{4cm}
      \includegraphics[width=\textwidth]{Spher_Abb1.png}
      %\begin{tiny}
      %\footcite{Egerton}}
      %\end{tiny}
    \end{figure}
    \end{column}
  \end{columns}
  \begin{columns}
    \begin{column}{0.6\textwidth}
    \vspace{-1cm}
    \begin{tiny}

        \begin{align}
          C_{s}&=\frac{e}{96m\tilde{U}}\int\left(\frac{2e}{m\tilde{U}}B_{z}^{4}+5\left(B_{z}'\right)^{2}-B_{z}B_{z}''\right)R^{4}dz \\
          &=\frac{e^{2}R^{4}}{4p_{z,0}^{2}}F_{3}+\frac{e^{4}R^{4}}{12p_{z,0}^{4}}F_{4}
        \end{align}
    \end{tiny}
    \end{column}
    \begin{column}{0.4\textwidth}
      \begin{figure}
      \caption{Illustration of beam radius change due to spherical aberration}
      \end{figure}
      \end{column}
  \end{columns}
  \end{frame}
